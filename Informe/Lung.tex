\documentclass[a4paper,12pt]{article}
\usepackage{booktabs}
\usepackage{amsmath}
\usepackage{hyperref}
\usepackage{geometry}
\usepackage{listings}
\usepackage{xcolor}
\usepackage{graphicx}

\lstset{
    language=Python,
    basicstyle=\ttfamily\small,
    keywordstyle=\color{blue},
    stringstyle=\color{red},
    commentstyle=\color{green},
    showstringspaces=false,
    numbers=left,
    numberstyle=\tiny\color{gray},
    breaklines=true,
    frame=single,
    captionpos=b
}

\geometry{margin=1in}

\title{Análisis de Datos del Cáncer de Pulmón NCCTG}
\author{}
\date{\today}

\begin{document}
	
	\maketitle
	
	\section*{Introducción}
	El cáncer de pulmón es una de las principales causas de mortalidad en todo el mundo, siendo responsable de millones de muertes cada año. En este contexto, los datos clínicos resultan fundamentales para entender los factores que afectan la supervivencia de los pacientes y, en última instancia, mejorar las estrategias de tratamiento. El conjunto de datos del cáncer de pulmón NCCTG proporciona información valiosa sobre pacientes con cáncer de pulmón avanzado recopilada por el Grupo de Tratamiento del Cáncer del Norte Central. Este conjunto de datos incluye tanto variables clínicas como datos autoinformados por los pacientes, lo que permite una visión integral de las condiciones de los mismos.
	
	El propósito de este análisis es explorar las características de este conjunto de datos, identificar patrones significativos y comprender cómo ciertas variables, como la edad, el sexo, las puntuaciones de rendimiento físico y la pérdida de peso, influyen en los tiempos de supervivencia. A través de este análisis, se espera contribuir a la identificación de posibles predictores de la supervivencia que podrían ser útiles en entornos clínicos y en investigaciones futuras.
	
	\section*{Descripción del Conjunto de Datos}
	El conjunto de datos contiene información detallada sobre 228 pacientes y se estructura en las siguientes variables clave:
	
	\begin{itemize}
		\item \textbf{inst}: Código de la institución donde se atendió al paciente. Esta variable permite identificar la procedencia de los datos.
		\item \textbf{time}: Tiempo de supervivencia del paciente en días desde el inicio del estudio hasta el momento del fallecimiento o censura.
		\item \textbf{status}: Estado de censura al final del estudio. Se codifica como 1 para censurado (es decir, el paciente estaba vivo al final del seguimiento) y 2 para fallecido.
		\item \textbf{age}: Edad del paciente en años al inicio del estudio. Esta variable es crucial para evaluar la influencia de la edad en la supervivencia.
		\item \textbf{sex}: Sexo del paciente, codificado como 1 para masculino y 2 para femenino.
		\item \textbf{ph.ecog}: Puntuación de rendimiento según el Índice de Rendimiento ECOG (Eastern Cooperative Oncology Group):
		\begin{itemize}
			\item 0: Asintomático.
			\item 1: Sintomático pero completamente ambulatorio.
			\item 2: En cama menos del 50\% del día.
			\item 3: En cama más del 50\% del día pero no postrado.
			\item 4: Postrado en cama.
		\end{itemize}
		\item \textbf{ph.karno}: Puntuación de rendimiento de Karnofsky asignada por el médico, que varía de 0 (muy mal estado) a 100 (excelente estado físico).
		\item \textbf{pat.karno}: Puntuación de rendimiento de Karnofsky asignada por el propio paciente, lo que permite comparar la percepción del paciente con la evaluación del médico.
		\item \textbf{meal.cal}: Cantidad de calorías consumidas durante las comidas diarias, una medida que puede reflejar el estado nutricional del paciente.
		\item \textbf{wt.loss}: Pérdida de peso en los últimos seis meses, medida en libras. Esta variable es un indicador importante del deterioro físico en pacientes con cáncer avanzado.
	\end{itemize}
	
	\section*{Contexto y Antecedentes}
	El uso de los datos clínicos para el análisis de supervivencia se ha convertido en una herramienta esencial en oncología. En particular, el conjunto de datos NCCTG fue recopilado para evaluar factores pronósticos relacionados con la supervivencia de pacientes con cáncer de pulmón avanzado. Una característica notable de este conjunto de datos es el uso de codificaciones específicas (por ejemplo, 1 y 2 para vivo y muerto, respectivamente) en lugar de las convencionales 0 y 1. Este enfoque, aunque inusual, se adoptó por motivos técnicos durante la era de las tarjetas perforadas en sistemas como el IBM 360 Fortran, donde los valores en blanco se interpretaban como ceros. Aunque estos sistemas han quedado obsoletos, las prácticas asociadas persistieron durante años.
	
	El estudio asociado con este conjunto de datos, liderado por Loprinzi et al. (1994), exploró cómo las puntuaciones de rendimiento y otras variables influían en la supervivencia. Estas puntuaciones, como las de Karnofsky y ECOG, han demostrado ser herramientas valiosas para evaluar el estado funcional de los pacientes y predecir resultados clínicos. Además, variables como la pérdida de peso y la ingesta calórica son indicadores del estado nutricional, que también juega un papel crítico en el manejo del cáncer.
	
	\section*{Objetivos del Análisis}
	Este informe tiene como objetivos:
	\begin{itemize}
		\item Resumir y describir las características principales de las variables del conjunto de datos.
		\item Identificar patrones y tendencias en los tiempos de supervivencia y los estados de censura.
		\item Evaluar la relación entre las puntuaciones de rendimiento, el estado nutricional y los resultados de supervivencia.
		\item Proporcionar una base para análisis futuros más complejos, como modelos de regresión de supervivencia.
	\end{itemize}
	
	\section*{Análisis de las variables}

	\subsection*{\underline{Inst}}

	Dado que este valor solo representa la institución donde se atendió al paciente, no se considera relevante para el análisis de supervivencia y, por lo tanto, no se incluirá en los análisis posteriores.

	\subsection*{\underline{Time}}

	La variable de tiempo de supervivencia es fundamental para el análisis de supervivencia y se presenta en días. \textbf{La Figura }\ref{fig:time_distribution} muestra un histograma de los tiempos de supervivencia, que revela una distribución asimétrica con una cola larga hacia la derecha. La mayoría de los pacientes sobrevivieron menos de 500 días, con un pico alrededor de los 200 días. La censura también es evidente en los datos, ya que hay un número significativo de pacientes cuyo tiempo de supervivencia no se conoce debido a la censura.
	
	\subsubsection*{Clasificación de la variable}

	En cuanto al tipo de variable es \textbf{cuantitativa discreta} debido a que se da en el número de días que sobrevivió el paciente y de \textbf{intervalos} ya que el 0 tiene un significado.


\subsubsection*{Medidas de tendencia central}

A continuación se presentan las medidas de tendencia central para la variable de tiempo de supervivencia:

\begin{table}[h!]
    \centering
    \begin{tabular}{|c|c|}
        \hline
        \textbf{Medida} & \textbf{Valor} \\
        \hline
        Media & 305.2325 \\
        \hline
        Moda & 163 \\
        \hline
        Mediana & 255 \\
        \hline
        Q1 (25\%) & 166 \\
        \hline
        Q3 (75\%) & 396 \\
        \hline
    \end{tabular}
    \caption{Medidas de tendencia central para el tiempo de supervivencia}
    \label{tab:medidas_tendencia_central}
\end{table}

\clearpage

\subsubsection*{Medidas de variabilidad}

A continuación se presentan las medidas de variabilidad para la variable de tiempo de supervivencia:

\begin{table}[h!]
    \centering
    \begin{tabular}{|c|c|}
        \hline
        \textbf{Medida} & \textbf{Valor} \\
        \hline
        Máximo & 1022 \\
        \hline
        Mínimo & 5 \\
        \hline
        Rango & 1017 \\
        \hline
        Varianza & 44371.54 \\
        \hline
        Desviación Estándar & 210.6455 \\
        \hline
        Coeficiente de Variación & 0.6901152 \\
        \hline
    \end{tabular}
    \caption{Medidas de variabilidad para el tiempo de supervivencia}
    \label{tab:medidas_variabilidad}
\end{table}

\subsubsection*{Prueba de Distribución}

El siguienete gráfico muestra la distribución de los tiempos de supervivencia comparada con diferentes tipos de distribución.

\begin{figure}[h]
 	\centering
 	\includegraphics[width=0.7\textwidth]{distribucion_time.png}
 	\caption{Distribución de los tiempos de supervivencia}
 	\label{fig:time_distribution}
\end{figure}

Por tanto es conveniente realizar las pruebas de hipótesis para determinar si la variable sigue una distribución específica. Dicha prueba resultó en los siguientes valores:

\newpage

\begin{table}[h!]
    \centering
    \begin{tabular}{|c|c|}
        \hline
        \textbf{Prueba} & \textbf{p-value} \\
        \hline
        Anderson-Darling (Normal) & 7.417945e-14 \\
        \hline
        Kolmogorov-Smirnov (Normal) & 0.001288452 \\
        \hline
        Shapiro-Wilk (Normal) & 5.114211e-10 \\
        \hline
        Kolmogorov-Smirnov (Exponencial) & 2.970448e-07 \\
        \hline
        Kolmogorov-Smirnov (Gamma) & 0.3606842 \\
        \hline
        Kolmogorov-Smirnov (Chi-cuadrado) & 3.916153e-192 \\
        \hline
    \end{tabular}
    \caption{Resultados de las pruebas de distribución para la variable \textit{time}}
    \label{tab:pruebas_distribucion}
\end{table}

Como es posible apreciar solo la prueba de Kolmogorov-Smirnov para la distribución Gamma no rechaza la hipótesis nula, por lo que se puede concluir que no es posible negar la variable \textbf{time} posee una distribución Gamma y por tanto la asumiremos en el análisis siguiente.

\subsubsection*{Intervalo de confianza para la media}

El intervalo de confianza para la media de la variable \textbf{time} es de 269.2987567105341 a 341.1661555701677 con un nivel de confianza del 99\%. Esto lo podemos saber gracias al siguiente código :

\begin{lstlisting}[language=Python, caption={Código en Python para calcular el intervalo de confianza},label={lst:intervalo_confianza_normal}]
    import pandas as pd
    import numpy as np
    from scipy.stats import norm as z
    
    # Leer el archivo CSV y seleccionar la columna 'time'
    muestra = pd.read_csv('lung_dataset.csv')['time'].dropna().values
    
    
    media_muestral = np.mean(muestra)
    desviacion_muestral = np.std(muestra, ddof=1)
    n = len(muestra)
    
    # Nivel de confianza y grados de libertad
    nivel_significancia = 0.01
    confianza = 1 - nivel_significancia
    

    z_critico = z.ppf(1 - nivel_significancia / 2)
    
    
    margen_error = z_critico * (desviacion_muestral / np.sqrt(n))
    limite_inferior = media_muestral - margen_error
    limite_superior = media_muestral + margen_error
    
    print(media_muestral, desviacion_muestral, (limite_inferior, limite_superior))
    \end{lstlisting}


    \subsubsection*{Prueba de hipótesis para la media}

    Se plantea que el tiempo de supervivencia de los pacientes con cáncer es de un promedio de 250 días. Probemos la veracidad de esta proposición mediante el siguiente código:

    \begin{lstlisting}[language=Python, caption={Código en Python para calcular el estadígrafo de la prueba de hipotesis},label={3st:prueba_hipotesis_time}]
        
    import math
    import pandas as pd
    from scipy.stats import t as t_dist

    muestra = pd.read_csv('lung_dataset.csv')['time'].dropna().values


    # H0: La media de la edad es <= 250
    # H1: La media de la edad es > 250
    mu_0 = 250  
    alpha = 0.01  # Nivel de significancia


    n = len(muestra)                     
    sample_mean = sum(muestra) / n       # Media muestral
    sample_std = math.sqrt(sum((x - sample_mean) ** 2 for x in muestra) / (n - 1))  
    t_stat = (sample_mean - mu_0) / (sample_std / math.sqrt(n))  

    # Grados de libertad
    df = n - 1


    t_critical = t_dist.ppf(1 - alpha, df)  

    \end{lstlisting}

    Este código nos da como resultado que se rechaza la hipótesis nula, por lo que el tiempo de supervivencia promedio de los pacientes con cáncer de pulmón es mayor a 250 días.

    \subsection*{\underline{Status}}

    La variable de estado de censura, codificada como 1 para censurado y 2 para fallecido, es crucial para el análisis de supervivencia. \textbf{La Figura }\ref{fig:status_distribution} muestra la distribución de los estados de censura en el conjunto de datos. La mayoría de los pacientes están censurados, lo que refleja la naturaleza de los estudios de supervivencia donde no todos los pacientes experimentan el evento de interés (es decir, la muerte) durante el período de seguimiento. 
    
    \subsubsection*{Clasificación de la variable}

    En cuanto al tipo de variable es \textbf{cualitativa nominal} debido a que se da en dos categorías y de \textbf{intervalos} ya que el 0 tiene un significado.

    \subsubsection*{Medidas de tendencia central}

A continuación se presentan las medidas de tendencia central para la variable de estado de censura:

\begin{table}[h!]
    \centering
    \begin{tabular}{|c|c|}
        \hline
        \textbf{Medida} & \textbf{Valor} \\
        \hline
        Media & 1.723684 \\
        \hline
        Moda & 2 \\
        \hline
        Mediana & 2 \\
        \hline
        Q1 (25\%) & 1 \\
        \hline
        Q3 (75\%) & 2 \\
        \hline
    \end{tabular}
    \caption{Medidas de tendencia central para la variable de estado de censura}
    \label{tab:medidas_tendencia_central_status}
\end{table}

\subsubsection*{Medidas de variabilidad}

A continuación se presentan las medidas de variabilidad para la variable de estado de censura:

\begin{table}[h!]
    \centering
    \begin{tabular}{|c|c|}
        \hline
        \textbf{Medida} & \textbf{Valor} \\
        \hline
        Máximo & 2 \\
        \hline
        Mínimo & 1 \\
        \hline
        Rango & 1 \\
        \hline
        Varianza & 0.200846 \\
        \hline
        Desviación Estándar & 0.448159 \\
        \hline
        Coeficiente de Variación & 0.260001 \\
        \hline
    \end{tabular}
    \caption{Medidas de variabilidad para la variable de estado de censura}
    \label{tab:medidas_variabilidad_status}
\end{table}

\newpage
\subsubsection*{Prueba de Distribución}

El siguiente gráfico muestra la distribución de los estados de censura en el conjunto de datos, este tipo de variable aleatoria es una Bernoulli ya que solo puede tomar dos valores, en este caso 1 y 2.

\begin{figure}[h]
    \centering
    \includegraphics[width=0.7\textwidth]{distribucion_status.png}
    \caption{Distribución de los tiempos de supervivencia}
    \label{fig:status_distribution}
\end{figure}

\subsubsection*{Intervalo de confianza para la proporción}

El intervalo de confianza para la proporción de la variable \textbf{status} es de 0.6474013349555908 a 0.7999670860970408 con un nivel de confianza del 99\%. Esto lo podemos saber gracias al siguiente código:

\begin{lstlisting}[language=Python, caption={Código en Python para calcular el intervalo de confianza},label={2st:intervalo_confianza_normal}]
    import pandas as pd
    import numpy as np
    from scipy.stats import norm as z
    
    # Leer el archivo CSV y seleccionar la columna 'status'
    muestra = pd.read_csv('lung_dataset.csv')['status'].dropna().values
    
    n = len(muestra)
    proporcion_fallecidos = np.sum(muestra == 2) / n
    
    # Nivel de confianza
    nivel_significancia = 0.01
    confianza = 1 - nivel_significancia
    
    z_critico = z.ppf(1 - nivel_significancia / 2)

    margen_error = z_critico * np.sqrt(proporcion_fallecidos * (1 - proporcion_fallecidos) / n)
    
    limite_inferior = proporcion_fallecidos - margen_error
    limite_superior = proporcion_fallecidos + margen_error

    print(f"Intervalo de confianza al {confianza*100}%: ({limite_inferior}, {limite_superior})")

\end{lstlisting}

    \subsection*{\underline{Age}}

    La edad del paciente al inicio del estudio es una variable importante para evaluar la influencia de la edad en la supervivencia. \textbf{La Figura }\ref{fig:age_distribution} muestra un histograma de las edades de los pacientes, que revela una distribución aproximadamente simétrica con un pico alrededor de los 65 años.
	
    \subsubsection*{Clasificación de la variable}

    En cuanto al tipo de variable es \textbf{cuantitativa discreta} debido a que se da en el número de años que tiene el paciente y de \textbf{intervalos} ya que el 0 tiene un significado.    

    \subsubsection*{Medidas de tendencia central}

    A continuación se presentan las medidas de tendencia central para la variable de edad:
    
    \begin{table}[h!]
        \centering
        \begin{tabular}{|c|c|}
            \hline
            \textbf{Medida} & \textbf{Valor} \\
            \hline
            Media & 62.44737 \\
            \hline
            Moda & 60 \\
            \hline
            Mediana & 63 \\
            \hline
            Q1 (25\%) & 56 \\
            \hline
            Q3 (75\%) & 69 \\
            \hline
        \end{tabular}
        \caption{Medidas de tendencia central para la variable de edad}
        \label{tab:medidas_tendencia_central_edad}
    \end{table}
    
    \newpage
    \subsubsection*{Medidas de variabilidad}
    
    A continuación se presentan las medidas de variabilidad para la variable de edad:
    
    \begin{table}[h!]
        \centering
        \begin{tabular}{|c|c|}
            \hline
            \textbf{Medida} & \textbf{Valor} \\
            \hline
            Máximo & 82 \\
            \hline
            Mínimo & 39 \\
            \hline
            Rango & 43 \\
            \hline
            Varianza & 82.3276142 \\
            \hline
            Desviación Estándar & 9.0734566 \\
            \hline
            Coeficiente de Variación & 0.1452977 \\
            \hline
        \end{tabular}
        \caption{Medidas de variabilidad para la variable de edad}
        \label{tab:medidas_variabilidad_edad}
    \end{table}

    \subsubsection*{Prueba de Distribución}

    El siguiente gráfico muestra la distribución de las edades de los pacientes comparada con diferentes tipos de distribución.

 \begin{figure}[h]
 	\centering
 	\includegraphics[width=0.7\textwidth]{distribucion_age.png}
 	\caption{Distribución de la edad de los pacientes}
 	\label{fig:age_distribution}
 \end{figure}

Por tanto es conveniente realizar las pruebas de hipótesis para determinar si la variable
sigue una distribución específica. Dicha prueba resultó en los siguientes valores:

\newpage 

\begin{table}[h!]
    \centering
    \begin{tabular}{|c|c|}
        \hline
        \textbf{Prueba} & \textbf{p-value} \\
        \hline
        Anderson-Darling (Normal) & 0.009280497 \\
        \hline
        Kolmogorov-Smirnov (Normal) & 0.2508968 \\
        \hline
        Shapiro-Wilk (Normal) & 0.00482907 \\
        \hline
        Kolmogorov-Smirnov (Exponencial) & 6.191832e-46 \\
        \hline
        Kolmogorov-Smirnov (Gamma) & 4.863935e-31 \\
        \hline
        Kolmogorov-Smirnov (Chi-cuadrado) & 1.831273e-198 \\
        \hline
    \end{tabular}
    \caption{Resultados de las pruebas de distribución para la variable \textit{age}}
    \label{tab:pruebas_distribucion_age}
\end{table}

    Como es posible apreciar bajo un nivel de significancia de 0.01 la prueba de Kolmogorov - Smirnov para la distribución normal no rechaza la hipótesis nula, por lo que se puede concluir que no es posible negar que la variable \textbf{age} posee una distribución normal y por tanto la  asumiremos en el análisis siguienete.
    
    \subsubsection*{Intervalo de confianza para la media}

    El intervalo de confianza para la media de la variable \textbf{age} es de 60.89954141062864 a 63.99519543147662 con un nivel de confianza del 99\%. Esto lo podemos saber gracias al siguiente código, al ser análogo al código \ref{lst:intervalo_confianza_normal} utilizado para el intervalo de confianza de la variable \textbf{time} solamente sustituiríamos:

    \begin{lstlisting}[language=Python, caption={Código en Python para calcular el intervalo de confianza}]
muestra = pd.read_csv('lung_dataset.csv')['age'].dropna().values
    
    \end{lstlisting}

    \subsubsection*{Prueba de hipótesis para la media}

    Se plantea que la edad promedio de los pacientes con cáncer es de más de 60 años. Utilizando la implementación análoga para la variable time \ref{3st:prueba_hipotesis_time}.

    \begin{lstlisting}[language=Python, caption={Código en Python para calcular el estimador de la prueba de hipotesis},label={3st:prueba_hipotesis_age}]
        

muestra = pd.read_csv('lung_dataset.csv')['age'].dropna().values

# H0: La media de la edad es <= 60
# H1: La media de la edad es > 60
mu_0 = 60  

    \end{lstlisting}

    Este código nos da como resultado que se rechaza la hipótesis nula, por lo que la edad promedio de los pacientes con cáncer de pulmón es más de 60 años. 

    \subsection*{\underline{Sex}}

    La variable de sexo del paciente, codificada como 1 para masculino y 2 para femenino, es un factor importante a considerar en el análisis de supervivencia.

    \subsubsection*{Clasificación de la variable}

    En cuanto al tipo de variable es \textbf{cualitativa nominal} debido a que se da en dos categorías y de \textbf{intervalos} ya que el 0 tiene un significado.

    \subsubsection*{Medidas de tendencia central}

A continuación se presentan las medidas de tendencia central para la variable \textit{sex}:

\begin{table}[h!]
    \centering
    \begin{tabular}{|c|c|}
        \hline
        \textbf{Medida} & \textbf{Valor} \\
        \hline
        Media & 1.394737 \\
        \hline
        Moda & 1 \\
        \hline
        Mediana & 1 \\
        \hline
        Q1 (25\%) & 1 \\
        \hline
        Q3 (75\%) & 2 \\
        \hline
    \end{tabular}
    \caption{Medidas de tendencia central para la variable \textit{sex}}
    \label{tab:medidas_tendencia_central_sex}
\end{table}

\subsubsection*{Medidas de variabilidad}

A continuación se presentan las medidas de variabilidad para la variable \textit{sex}:

\begin{table}[h!]
    \centering
    \begin{tabular}{|c|c|}
        \hline
        \textbf{Medida} & \textbf{Valor} \\
        \hline
        Máximo & 2 \\
        \hline
        Mínimo & 1 \\
        \hline
        Rango & 1 \\
        \hline
        Varianza & 0.2399722 \\
        \hline
        Desviación Estándar & 0.4898696 \\
        \hline
        Coeficiente de Variación & 0.3512272 \\
        \hline
    \end{tabular}
    \caption{Medidas de variabilidad para la variable \textit{sex}}
    \label{tab:medidas_variabilidad_sex}
\end{table}
    
 \newpage

    \subsubsection*{Prueba de Distribución}

    \textbf{La Figura }\ref{fig:sex_distribution} muestra la distribución de los pacientes por sexo. Como se puede observar, hay una proporción ligeramente mayor de pacientes masculinos en comparación con los femeninos. Esta diferencia en la distribución de sexos puede influir en los resultados de supervivencia y, por lo tanto, es crucial tenerla en cuenta en cualquier análisis posterior.

    \begin{figure}[h]
        \centering
        \includegraphics[width=0.7\textwidth]{distribucion_sex.png}
        \caption{Distribución del género de los pacientes}
        \label{fig:sex_distribution}
    \end{figure}

    \subsubsection*{Intervalo de confianza para la proporción}

    El intervalo de confianza para la proporción de la variable \textbf{sex} es de 0.31135420869499797 a 0.47811947551552836 con un nivel de confianza del 99\%. Esto lo podemos saber gracias al siguiente código, al ser análogo al código \ref{2st:intervalo_confianza_normal} utilizado para el intervalo de confianza de la variable \textbf{status} solamente sustituiríamos:

    \begin{lstlisting}[language=Python, caption={Código en Python para calcular el intervalo de confianza}]
        muestra = pd.read_csv('lung_dataset.csv')['sex'].dropna().values

    \end{lstlisting}

    \subsubsection*{Prueba de hipótesis para dos muestras}

    Verificación de la existencia de una diferencia significativa en la proporción de personas fallecidas (status = 2) entre los géneros (sex: 1 para hombres, 2 para mujeres). Esto permitiría explorar si el género está asociado con la probabilidad de fallecimiento en el contexto del estudio. Nos apoyaremos en el siguiente código para darle solución a la proposición.

    \begin{lstlisting}[language=Python]
        
        import math
from scipy.stats import norm
import pandas as pd

# Datos del dataset
# Status = 2 (Fallecidos)
# Sex = 1 (Hombres), 2 (Mujeres)

# Leer el archivo CSV y seleccionar las columnas necesarias
lung_data = pd.read_csv('lung_dataset.csv')[['sex', 'status']].dropna()

# Contar fallecidos y totales para hombres y mujeres
male_fallecidos = lung_data[(lung_data['sex'] == 1) & (lung_data['status'] == 2)].shape[0]
female_fallecidos = lung_data[(lung_data['sex'] == 2) & (lung_data['status'] == 2)].shape[0]
male_total = lung_data[lung_data['sex'] == 1].shape[0]
female_total = lung_data[lung_data['sex'] == 2].shape[0]

# Proporciones de fallecidos para hombres y mujeres
p_male = male_fallecidos / male_total
p_female = female_fallecidos / female_total

p_pool = (male_fallecidos + female_fallecidos) / (male_total + female_total)

z_stat = (p_male - p_female) / math.sqrt(
    p_pool * (1 - p_pool) * (1 / male_total + 1 / female_total)
)

alpha = 0.05
z_critical = norm.ppf(1 - alpha / 2)


    \end{lstlisting}

    Después de obtener los resultados del código vemos que se rechaza la hipótesis nula, por lo que hay diferencias significativas en la mortalidad entre sexos, lo cual podría tener implicaciones importantes para la investigación futura y las políticas de salud pública.

    \subsection*{\underline{Pat.karno}}

    La puntuación de rendimiento de Karnofsky asignada por el propio paciente es una medida importante del estado funcional percibido por el paciente. \textbf{La Figura }\ref{fig:pat.karno} muestra un histograma de las puntuaciones de rendimiento de Karnofsky asignadas por los pacientes, que revela una distribución asimétrica con un pico alrededor de 80-90. Las puntuaciones de rendimiento de Karnofsky asignadas por los pacientes pueden diferir de las asignadas por los médicos y pueden proporcionar información adicional sobre la percepción del paciente sobre su estado funcional.

    \subsubsection*{Clasificación de la variable}

    En cuanto al tipo de variable es \textbf{cuantitativa discreta} debido a que se da en el número de puntos asignados por el paciente de 0 a 100 y de \textbf{intervalos} ya que el 0 tiene un significado.

    \subsubsection*{Medidas de tendencia central}

    A continuación se presentan las medidas de tendencia central para la variable \textit{pat.karno}:
    
    \begin{table}[h!]
        \centering
        \begin{tabular}{|c|c|}
            \hline
            \textbf{Medida} & \textbf{Valor} \\
            \hline
            Media & 79.95556 \\
            \hline
            Moda & 90 \\
            \hline
            Mediana & 80 \\
            \hline
            Q1 (25\%) & 70 \\
            \hline
            Q3 (75\%) & 90 \\
            \hline
        \end{tabular}
        \caption{Medidas de tendencia central para la variable \textit{pat.karno}}
        \label{tab:medidas_tendencia_central_pat_karno}
    \end{table}

    \subsubsection*{Medidas de variabilidad}
    
    A continuación se presentan las medidas de variabilidad para la variable \textit{pat.karno}:
    
    \begin{table}[h!]
        \centering
        \begin{tabular}{|c|c|}
            \hline
            \textbf{Medida} & \textbf{Valor} \\
            \hline
            Máximo & 100 \\
            \hline
            Mínimo & 30 \\
            \hline
            Rango & 70 \\
            \hline
            Varianza & 213.8373016 \\
            \hline
            Desviación Estándar & 14.6231769 \\
            \hline
            Coeficiente de Variación & 0.1828913 \\
            \hline
        \end{tabular}
        \caption{Medidas de variabilidad para la variable \textit{pat.karno}}
        \label{tab:medidas_variabilidad_pat_karno}
    \end{table}

    \newpage

    \subsubsection*{Prueba de Distribución}

    El siguiente gráfico muestra la distribución de las puntuaciones de rendimiento de Karnofsky asignadas por los pacientes comparada con diferentes tipos de distribución.

    \begin{figure}[h]
        \centering
        \includegraphics[width=0.7\textwidth]{distribucion_pat.karno.png}
        \caption{Distribución de las puntuaciones de rendimiento de Karnofsky asignadas por los pacientes}
        \label{fig:pat.karno}
    \end{figure}


    \subsubsection*{Intervalo de confianza para la media}

    \subsubsection*{Prueba de hipótesis para la media}

    Se plantea que la puntuación promedio asignada por los pacientes con cáncer de pulmón, en una escala de 1 a 100, en función de su bienestar corporal es de 75. Utilizando la implementación análoga para la variable time  \ref{3st:prueba_hipotesis_time}.

    \begin{lstlisting}[language=Python, caption={Código en Python para calcular el estimador de la prueba de hipotesis},label={3st:prueba_hipotesis_pat.karno}]
        
muestra = pd.read_csv('lung_dataset.csv')['pat.karno'].dropna().values

# H0: La media de la edad es <= 75
# H1: La media de la edad es > 75
mu_0 = 75  

    \end{lstlisting}

 El resultado de este código nos dice que se rechaza la hipótesis nula, por lo que la calificación que dada por cada paciente es superior a 75, lo que nos dice que en promedio los pacientes sienten mejoría ante su estado de enfermedad.

    \subsection*{\underline{Wt.loss}}

    La variable \texttt{wt.loss} representa la pérdida de peso en los pacientes con cáncer de pulmón. Específicamente, esta variable indica la cantidad de peso que ha perdido un paciente en los últimos seis meses antes de ser diagnosticado o evaluado.

La pérdida de peso es un indicador importante en oncología, ya que puede reflejar el estado nutricional del paciente y la progresión de la enfermedad. En muchos casos, una pérdida significativa de peso puede asociarse con un pronóstico más desfavorable y puede influir en las decisiones sobre el tratamiento y el manejo del paciente.

Es importante considerar esta variable junto con otras características clínicas y demográficas para obtener una comprensión más completa del estado del paciente y su supervivencia.
Esta variable tiene valores negativos, lo que indica que algunos pacientes ganaron peso en lugar de perderlo. Esto puede deberse a una variedad de factores, como la retención de líquidos, la ganancia de peso relacionada con el tratamiento o la variabilidad en la medición del peso. En cualquier caso, es importante tener en cuenta esta variabilidad al interpretar los resultados del análisis.

\subsubsection*{Clasificación de la variable}

En cuanto al tipo de variable es \textbf{cuantitativa discreta}, aunque el peso generalmente se toma como una variable continua, en este caso se hace un redondeo a los \textbf{enteros} que representa el número de libras que ha perdido el paciente. Es de \textbf{intervalos} ya que el 0 tiene un significado.

\subsubsection*{Medidas de tendencia central}

A continuación se presentan las medidas de tendencia central para la variable \textit{wt.loss}:

\begin{table}[h!]
    \centering
    \begin{tabular}{|c|c|}
        \hline
        \textbf{Medida} & \textbf{Valor} \\
        \hline
        Media & 9.831776 \\
        \hline
        Moda & 0.000000 \\
        \hline
        Mediana & 7.000000 \\
        \hline
        Q1 (25\%) & 0.000000 \\
        \hline
        Q3 (75\%) & 15.750000 \\
        \hline
    \end{tabular}
    \caption{Medidas de tendencia central para la variable \textit{wt.loss}}
    \label{tab:medidas_tendencia_central_wt_loss}
\end{table}

\subsubsection*{Medidas de variabilidad}

A continuación se presentan las medidas de variabilidad para la variable \textit{wt.loss}:

\begin{table}[h!]
    \centering
    \begin{tabular}{|c|c|}
        \hline
        \textbf{Medida} & \textbf{Valor} \\
        \hline
        Máximo & 68.000000 \\
        \hline
        Mínimo & -24.000000 \\
        \hline
        Rango & 92.000000 \\
        \hline
        Varianza & 172.657014 \\
        \hline
        Desviación Estándar & 13.139902 \\
        \hline
        Coeficiente de Variación & 1.336473 \\
        \hline
    \end{tabular}
    \caption{Medidas de variabilidad para la variable \textit{wt.loss}}
    \label{tab:medidas_variabilidad_wt_loss}
\end{table}

\subsubsection*{Prueba de Distribución}

El siguiente gráfico muestra la distribución de la pérdida de peso en los pacientes comparada con diferentes tipos de distribución.

\begin{figure}[h]
    \centering
    \includegraphics[width=0.7\textwidth]{distribución_wt_loss.png}
    \caption{Distribución de la pérdida(o ganancia) de peso}
    \label{fig:distribución_wt_loss}
\end{figure}

\newpage

Como es posible notar la distribución empírica de la pérdida de peso no sigue ninguna de las distribuciones conocidas por lo que realizaremos un análisis diferenciado entre las personas que perdieron peso y las que ganaron peso.
Haciendo dicho análisis podemos diferenciar los siguientes gráficos:

\begin{figure}[h]
    \centering
    \includegraphics[width=0.7\textwidth]{distribución_wt_loss_perdida.png}
    \caption{Distribución de la pérdida de peso}
    \label{fig:distribución_wt_loss_perdida}
\end{figure}

\begin{figure}[h]
    \centering
    \includegraphics[width=0.7\textwidth]{distribución_wt_loss_ganancia.png}
    \caption{Distribución de la ganancia de peso}
    \label{fig:distribución_wt_loss_ganancia}
\end{figure}

\newpage

Por lo tanto, es conveniente realizar las pruebas de hipótesis para determinar si las variables siguen una distribución específica. Dicha prueba resultó en los siguientes valores:

\begin{table}[h!]
    \centering
    \begin{tabular}{|c|c|}
        \hline
        \textbf{Prueba} & \textbf{p-value} \\
        \hline
        Anderson-Darling (Normal) & 1.197951e-11 \\
        \hline
        Kolmogorov-Smirnov (Normal) & 0.005456181 \\
        \hline
        Shapiro-Wilk (Normal) & 4.066897e-10 \\
        \hline
        Kolmogorov-Smirnov (Exponencial) & 0.09287115 \\
        \hline
        Kolmogorov-Smirnov (Gamma) & 0.03941509 \\
        \hline
        Kolmogorov-Smirnov (Chi-cuadrado) & 7.692063e-54 \\
        \hline
    \end{tabular}
    \caption{Resultados de las pruebas de distribución para la variable \textit{wt.loss} de los que perdieron peso}
    \label{tab:pruebas_distribucion_wt_loss_PERDIDA}
\end{table}

\begin{table}[h!]
    \centering
    \begin{tabular}{|c|c|}
        \hline
        \textbf{Prueba} & \textbf{p-value} \\
        \hline
        Anderson-Darling (Normal) & 1.613223e-05 \\
        \hline
        Kolmogorov-Smirnov (Normal) & 0.06862226 \\
        \hline
        Shapiro-Wilk (Normal) & 6.581438e-05 \\
        \hline
        Kolmogorov-Smirnov (Exponencial) & 0.4796857 \\
        \hline
        Kolmogorov-Smirnov (Gamma) & 0.03938671 \\
        \hline
        Kolmogorov-Smirnov (Chi-cuadrado) & 0.03611553 \\
        \hline
    \end{tabular}
    \caption{Resultados de las pruebas de distribución para la variable \textit{wt.loss} de los que ganaron peso}
    \label{tab:pruebas_distribucion_wt_loss_GANANCIA}
\end{table}

Por lo que podemos concluir que tanto la distribución de la pérdida de peso como la de la ganancia de peso siguen una distribución Exponencial.

\subsubsection*{Intervalo de confianza para las media de las personas que perdieron peso}

El intervalo de confianza para la media de la variable \textbf{wt.loss} de las personas que perdieron peso es de 12.086570460717061 a 18.346939910801844 con un nivel de confianza del 99\%. Esto lo podemos saber gracias al siguiente código:

\begin{lstlisting}[language=Python, caption={Código en Python para calcular el intervalo de confianza para la media de una distribución exponencial}, label={lst:intervalo_confianza_exponencial}]
    import pandas as pd
    import numpy as np
    from scipy.stats import chi2

    # Leer el archivo CSV y seleccionar la columna 'wt.loss'
    df = pd.read_csv('lung_dataset.csv')
    wt_loss = df['wt.loss'].dropna()

    # Filtrar los valores mayores que 0
    wt_loss_filtered = wt_loss[wt_loss > 0].values

    # Parametros
    n = len(wt_loss_filtered)
    alpha = 0.01  # Nivel de significancia (para un intervalo de confianza del 99%)

    # Suma de los datos
    sum_data = np.sum(wt_loss_filtered)

    # Valores criticos de chi-cuadrado
    chi2_lower = chi2.ppf(alpha / 2, 2 * n)
    chi2_upper = chi2.ppf(1 - alpha / 2, 2 * n)

    # Intervalo de confianza
    lower_bound =  (2 * sum_data)/chi2_upper 
    upper_bound = (2 * sum_data)/chi2_lower

    print(f"Intervalo de confianza para la media de una distribucion exponencial: ({lower_bound}, {upper_bound})")

\end{lstlisting}

\subsubsection*{Intervalo de confianza para la media de la personas que ganaron peso}

Análogamente el intervalo de confianza para las media de las personas que ganaron peso está entre 3.621220157012075 y 9.87694081167703 con un 99\% de confianza.

\subsubsection*{Prueba de Hipótesis}

El cáncer de pulmón avanzado es uno de los que estadísticamente provoca más perdida de peso en aquellos que lo sufren. Según estudios, perder más del 5\% del peso corporal en 6 meses o menos es un indicador de un deterioro físico avanzado, dado que el promedio de masa corporal humana a nivel mundial es de 137 lbs, probaremos que aquellos pacientes que fallecen tienden a tener una perdida de peso corporal de 6.85 lbs (5\% de 137 lbs).

Tomemos como hipótesis nula: El valor medio de la perdida de peso en 6 meses de los pacientes que fallecieron durante el estudio es menor de 6.85 lbs.

Por tanto la hipótesis alternativa es que dicho valor es mayor que 6.85 lbs.

Esto lo podemos calcular con el siguiente código, para un nivel de significación del 0.01

\begin{lstlisting}[language=Python]
import numpy as np
import pandas as pd
from scipy.stats import chi2

valor_hipotetico = 6.85
# Leer el archivo CSV y seleccionar la columna 'wt.loss' y 'status'
df = pd.read_csv('lung_dataset.csv')
wt_loss = df[df['status'] == 2]['wt.loss'].dropna()

# Filtrar los valores mayores que 0
wt_loss_filtered = wt_loss[wt_loss > 0].values

def prueba_hipotesis_exponencial(datos, media_poblacional, alfa=0.05):
    """
    Realiza una prueba de hipotesis para verificar si la media poblacional
    es menor que un valor dado en una distribucian exponencial.

    Parametros:
    - datos: lista o array de datos muestrales.
    - media_poblacional: valor de media poblacional bajo la hipotesis nula.
    - alfa: nivel de significancia (por defecto 0.05).

    Retorna:
    - Resultado de la prueba y el valor p.
    """
    # Calculo de la media muestral y el tamano de la muestra
    media_muestral = np.mean(datos)
    n = len(datos)
    
    # Hipotesis:
    # H0: Media poblacional <= media_poblacional
    # H1: Media poblacional > media_poblacional (prueba unilateral)

    # Varianza muestral para una exponencial (estimador de maxima verosimilitud)
    varianza_muestral = np.var(datos, ddof=1)
    
    # Estadistico de prueba (estimador 1/x)
    estadistico_prueba = n * (media_muestral / media_poblacional)
    #print(estadistico_prueba)
    # Valor critico y comparacion
    valor_critico = chi2.ppf(1 - alfa, df=n)
    #print(valor_critico)
    p_valor = 1 - chi2.cdf(estadistico_prueba, df=n)

    # Resultado
    if estadistico_prueba > valor_critico:
        resultado = "Rechazamos la hipotesis nula"
    else:
        resultado = "No podemos rechazar la hipotesis nula"
    
    return resultado, p_valor, estadistico_prueba

resultado, p_valor, estadistico = prueba_hipotesis_exponencial(wt_loss_filtered, 6.85, 0.01)
print(f"Resultado: {resultado}")
print(f"Valor p: {p_valor}")
print(f"Estadistico de prueba: {estadistico}")

\end{lstlisting}

Bajo esta prueba se rechaza la hipótesis nula bajo un p-value de 4.779732165616224e-12 por lo que podemos decir que podemos asumir la hipótesis alternativa con un 1\% de significancia, y por tanto estadísticamente las personas que fallecen presentan una disminución de peso de más de 6.85 lbs
\subsection*{Análisis de Correlación}

Para entender mejor las relaciones entre las variables numéricas en nuestro conjunto de datos, realizamos un análisis de correlación utilizando dos métodos diferentes: el coeficiente de correlación de Pearson y el coeficiente de correlación de Spearman.

\subsubsection*{Metodología}
Se utilizaron dos matrices de correlación:
\begin{itemize}
    \item \textbf{Correlación de Pearson}: Mide la fuerza y dirección de la relación lineal entre variables.
    \item \textbf{Correlación de Spearman}: Evalúa la relación monótona entre variables, siendo más robusta a valores atípicos.
\end{itemize}

\begin{figure}[h]
    \centering
    \includegraphics[width=0.45\textwidth]{correlacion_pearson.png}
    \caption{Matriz de correlación de Pearson}
    \label{fig:correlacion_pearson}
\end{figure}

\newpage
\subsubsection*{Correlaciones Detalladas}

\begin{table}[h!]
    \centering
    \small
    \begin{tabular}{|l|r|}
        \hline
        \textbf{Par de Variables} & \textbf{Correlación de Pearson} \\
        \hline
        inst - time & 0.02 \\
        inst - status & -0.13 \\
        inst - age & 0.05 \\
        inst - sex & 0.08 \\
        inst - ph.ecog & 0.06 \\
        inst - ph.karno & -0.02 \\
        inst - pat.karno & 0.04 \\
        inst - meal.cal & 0.10 \\
        inst - wt.loss & -0.17 \\
        time - status & -0.16 \\
        time - age & -0.08 \\
        time - sex & 0.11 \\
        time - ph.ecog & -0.19 \\
        time - ph.karno & 0.09 \\
        time - pat.karno & 0.18 \\
        time - meal.cal & 0.07 \\
        time - wt.loss & 0.03 \\
        status - age & 0.16 \\
        status - sex & -0.22 \\
        status - ph.ecog & 0.24 \\
        status - ph.karno & -0.16 \\
        status - pat.karno & -0.19 \\
        status - meal.cal & 0.02 \\
        status - wt.loss & 0.05 \\
        \hline
    \end{tabular}
    \caption{Correlaciones de Pearson (Parte 1)}
    \label{tab:correlaciones_pearson1}
\end{table}

\begin{table}[h!]
    \centering
    \small
    \begin{tabular}{|l|r|}
        \hline
        \textbf{Par de Variables} & \textbf{Correlación de Spearman} \\
        \hline
        ph.ecog - ph.karno & -0.84 \\
        ph.ecog - pat.karno & -0.53 \\
        ph.karno - pat.karno & 0.51 \\
        age - ph.ecog & 0.32 \\
        age - ph.karno & -0.33 \\
        age - pat.karno & -0.24 \\
        status - ph.ecog & 0.24 \\
        time - ph.ecog & -0.24 \\
        status - sex & -0.22 \\
        pat.karno - wt.loss & -0.22 \\
        meal.cal - wt.loss & -0.25 \\
        time - pat.karno & 0.21 \\
        time - ph.karno & 0.19 \\
        sex - wt.loss & -0.19 \\
        ph.karno - wt.loss & -0.20 \\
        \hline
    \end{tabular}
    \caption{Correlaciones de Spearman más relevantes (|r| > 0.15)}
    \label{tab:correlaciones_spearman}
\end{table}

\subsubsection*{Hallazgos Principales}
Las correlaciones más significativas (con umbral de 0.7) encontradas fueron:

\begin{itemize}
    \item \textbf{ph.karno y pat.karno}: Correlación de 0.72. Esto indica una fuerte relación positiva entre las evaluaciones de rendimiento realizadas por médicos y pacientes. Sugiere consistencia entre las percepciones de los profesionales y los pacientes sobre el estado funcional.
    \item La correlación negativa más fuerte se observa entre \textbf{ph.ecog} y \textbf{ph.karno} (r = -0.84), lo cual es esperado ya que miden aspectos similares pero en escalas opuestas.
    \item El \textbf{status} muestra correlaciones moderadas negativas con \textbf{sex} y positivas con \textbf{ph.ecog}, sugiriendo diferencias en la supervivencia entre géneros.
    \item La \textbf{edad} muestra correlaciones moderadas con las medidas de rendimiento (\textbf{ph.ecog}, \textbf{ph.karno}, \textbf{pat.karno}), indicando que el estado funcional tiende a deteriorarse con la edad.
\end{itemize}

Esta correlación tiene sentido desde una perspectiva clínica, ya que ambas medidas evalúan el estado funcional del paciente, aunque desde diferentes perspectivas. La moderada pero no perfecta correlación sugiere que, si bien hay acuerdo general entre las evaluaciones de médicos y pacientes, también hay diferencias importantes en las percepciones que podrían ser clínicamente relevantes.
\subsection*{Análisis de Regresión}


El análisis de regresión con un modelo OLS en este caso no es factible por ciertas razones expuestas posteriormente, por tanto proponemos dicho análisis utilizando una técnica conocida, el modelo Weibull AFT. \\
La regresión con modelo Weibull AFT (Accelerated Failure Time) es una técnica utilizada en análisis de supervivencia para modelar el tiempo hasta que ocurre un evento de interés, como la muerte o la falla de un componente.
 En este contexto, el modelo Weibull AFT asume que el tiempo de supervivencia se puede modelar como una función de los efectos de las covariables y que sigue una distribución Weibull.\\

¿En qué consiste un modelo Weibull AFT?

1. \textbf{Estructura del modelo}: En un modelo AFT, se supone que el tiempo hasta el evento ($T$) está relacionado con un vector de covariables ($X$) a través de una relación multiplicativa. Específicamente, se puede expresar como:
   \[
   T = \exp(X\beta) \cdot Z
   \]
   donde $Z$ es una variable aleatoria que sigue una distribución Weibull.

2. \textbf{Distribución Weibull}: La función de supervivencia para la distribución Weibull se define como:
   \[
   S(t) = e^{-(t/\lambda)^\kappa}
   \]
   donde $\lambda$ es el parámetro de escala y $\kappa$ es el parámetro de forma. Dependiendo del valor de $\kappa$, la distribución puede ser monótonamente creciente o decreciente.

3. \textbf{Interpretación}: En el contexto de la regresión AFT, los coeficientes estimados ($\beta$) indican cómo las covariables afectan el tiempo hasta el evento. Un coeficiente positivo implica que un aumento en la covariable está asociado con un aumento en el tiempo hasta el evento (es decir, una vida útil más larga).

¿Por qué usar Weibull AFT en vez de OLS?

1. \textbf{Naturaleza del dato}: OLS (Mínimos Cuadrados Ordinarios) asume que los errores son normalmente distribuidos y que la relación entre las variables es lineal. Sin embargo, en análisis de supervivencia, los datos suelen estar censurados (es decir, no todos los individuos experimentan el evento antes de que termine el estudio), lo que hace que OLS no sea apropiado.

2. \textbf{Censura}: El modelo AFT maneja adecuadamente la censura, permitiendo que se utilicen todos los datos disponibles, incluso aquellos para los cuales no se ha observado el evento. OLS no puede manejar la censura correctamente, lo que puede llevar a estimaciones sesgadas.

3. \textbf{Distribución del tiempo}: El modelo Weibull AFT permite modelar explicitamente la distribución del tiempo hasta el evento, lo cual es fundamental en análisis de supervivencia. OLS no tiene en cuenta la naturaleza del tiempo hasta el evento y puede dar resultados engañosos.

4. \textbf{Interpretación más adecuada}: Los modelos AFT proporcionan interpretaciones más intuitivas en términos del tiempo hasta un evento, lo que es especialmente útil en contextos clínicos o industriales donde el tiempo es una variable crítica.

En resumen, el uso del modelo Weibull AFT es preferido en situaciones donde se está analizando el tiempo hasta un evento y se presentan datos censurados, mientras que OLS no es adecuado debido a sus supuestos y limitaciones en este contexto.


\subsubsection*{Fórmula del Modelo}
El modelo Weibull AFT se plantea como:
\begin{equation}
\ln(T) = \beta_0 + \beta_1 \cdot \text{age} + \beta_2 \cdot \text{sex} +  \beta_3 \cdot \text{pat.karno} + \beta_4 \cdot \text{wt.loss} + \sigma \epsilon
\end{equation}

\subsubsection*{Componentes}
\begin{itemize}
    \item \textbf{Variables del dataset}:
    \begin{itemize}
        \item $T$: \texttt{time} (tiempo de supervivencia en días)
        \item \texttt{status}: Indicador de censura (1 = evento observado, 2 = censurado)
        \item \texttt{age}: Edad en años
        \item \texttt{sex}: Género (1 = hombre, 0 = mujer) (Se realizó un ajuste para el modelo)
        \item \texttt{pat.karno}: Puntuación Karnofsky (0-100)
        \item \texttt{wt.loss}: Pérdida de peso (lbs)
    \end{itemize}
    
    \item \textbf{Parámetros}:
    \begin{itemize}
        \item $\beta_0$: Intercepto
        \item $\beta_1, \dots, \beta_4$: Coeficientes de regresión
        \item $\sigma$: Parámetro de escala Weibull
        \item $\epsilon$: Error con distribución de valor extremo (extreme value para Weibull)
    \end{itemize}
\end{itemize}

\subsubsection*{Fórmula del Modelo Ajustado}

Luego de analizar los p valores (descartar aquellos mayores a 0.05, no son significativos para el modelo) el modelo queda:

\begin{equation}
    \ln(T) = \beta_0 + \beta_1 \cdot \text{sex} +  \beta_2 \cdot \text{pat.karno}  + \sigma \epsilon
\end{equation}

\subsubsection*{Resultados de la Regresión}
\begin{tabular}{lrrrr}
\toprule
Predictor & Coeficiente ($\beta$) & Error Estándar & $z$-value & $p$-value \\
\midrule
Intercepto & 5.16967 & 0.37717 & 13.71 & $<2 \times 10^{-16}$ \\
sex (Hombre=1) & -0.33580 & 0.14435 & -2.33 & 0.0200 \\
pat.karno & 0.01373 & 0.00461 & 2.98 & 0.0029 \\
Log(scale) & -0.31650 & 0.07224 & -4.38 & $1.2 \times 10^{-5}$ \\
\bottomrule
\end{tabular}

\vspace{1cm}
\textbf{Escala (scale)}: 0.729 \\
\textbf{Distribución}: Weibull \\

\subsubsection*{Interpretación Clave}
Efectos de los Predictores
\begin{itemize}
\item \textbf{Sexo ($\beta_{\text{sex}} = -0.336$)}:
\begin{equation*}
\exp(-0.336) \approx 0.714 \implies \text{Hombres tienen 28.6\% menos tiempo de supervivencia que mujeres}
\end{equation*}
\textit{Ejemplo}: Si una mujer tiene un tiempo de supervivencia estimado de 200 días, un hombre comparable tendría un tiempo de supervivencia estimado de $200 \times 0.714 \approx 143$ días. Esto sugiere que el género masculino está asociado con una menor supervivencia en comparación con el género femenino en este conjunto de datos.

\item \textbf{Puntuación Karnofsky ($\beta_{\text{pat.karno}} = 0.0137$)}:
\begin{equation*}
\exp(0.0137) \approx 1.0138 \implies \text{Cada punto adicional en la puntuación Karnofsky aumenta la supervivencia en 1.38\%}
\end{equation*}
\textit{Ejemplo}: Si un paciente mejora su puntuación Karnofsky en 10 puntos, su tiempo de supervivencia se incrementaría en aproximadamente un 14.6\%. Esto indica que una mejor percepción del estado funcional por parte del paciente está asociada con una mayor supervivencia.
\end{itemize}

\subsubsection*{Parámetro de Forma Weibull}
El parámetro de forma de la distribución Weibull es crucial para entender la naturaleza del riesgo a lo largo del tiempo. En este caso:
\begin{equation*}
\alpha = \frac{1}{\text{scale}} = \frac{1}{0.729} \approx 1.372 \quad (\textit{Riesgo creciente con el tiempo})
\end{equation*}
Un parámetro de forma $\alpha > 1$ indica que el riesgo de fallecimiento aumenta con el tiempo. Esto es consistente con la progresión de enfermedades como el cáncer de pulmón, donde el riesgo de mortalidad tiende a incrementarse a medida que la enfermedad avanza.

\subsubsection*{Conclusiones del Modelo}
El análisis de regresión con el modelo Weibull AFT ha revelado varios hallazgos importantes:
\begin{itemize}
\item El género masculino está asociado con una menor supervivencia en comparación con el género femenino.
\item Una mejor puntuación Karnofsky, que refleja una mejor percepción del estado funcional por parte del paciente, está asociada con una mayor supervivencia.
\item El riesgo de fallecimiento aumenta con el tiempo, lo cual es consistente con la naturaleza progresiva del cáncer de pulmón avanzado.
\end{itemize}
Estos resultados subrayan la importancia de considerar tanto las características demográficas como las percepciones del estado funcional en el análisis de supervivencia de pacientes con cáncer de pulmón. Además, el uso del modelo Weibull AFT ha permitido manejar adecuadamente la censura en los datos y proporcionar una interpretación más precisa de los factores que influyen en la supervivencia.

\section*{Conclusiones}
El conjunto de datos del cáncer de pulmón NCCTG proporciona información valiosa sobre los resultados de supervivencia en pacientes con cáncer de pulmón avanzado. Los datos destacan la importancia de las puntuaciones de rendimiento y la pérdida de peso como posibles predictores de la supervivencia. Los estudios futuros podrían utilizar este conjunto de datos para modelado pronóstico más profundo y validación de herramientas clínicas.

\section*{Referencias}
\begin{itemize}
    \item Loprinzi CL, Laurie JA, Wieand HS, Krook JE, Novotny PJ, Kugler JW, et al. "Evaluación prospectiva de variables pronósticas a partir de cuestionarios completados por los pacientes." \textit{Journal of Clinical Oncology}. 12(3):601-7, 1994.
    \item Therneau T. Documentación del paquete \textit{survival}.
\end{itemize}
	
\end{document}
